\documentclass[a4paper]{report}

\begin{document}
\title{LIFE}
\author{KAI YIN, CHAN}
\date{Since August 23, 2017}
\maketitle

\chapter{Purpose}
Purpose, according to Oxford Dictionary, is defined as ``the reason for which something is done or created or for which something exists.'' This chapter will, therefore, tell the readers the reason for which ``LIFE'' is written.

\chapter{Money}
\section{How Money is Created}
It starts with government creating IOUs, for example, bonds and treasury notes. They are a promise to pay a specified sum at a specified interest on a specified date. It is a debt owned by the government. In principle, government has created cash, but bonds and treasury notes do not look like cash yet.

The Federal Reserve System then acts as a money printing press to convert the IOUs created by the government to cash. It is called Federal Reserve Check. The Federal Reserve System takes government's IOUs as assets and issues Federal Reserve Check as a liability. The accounting book is completely balanced.

The government therefore received cash (Federal Reserve Check) and the Federal Reserve Check becomes government deposit. It is used to pay government expenses and, thus, is transformed to many government check. Up to this point, the accounting book of the government is completely balanced with Federal Reserve Check as asset and IOUs as liability.

The Federal Reserve Check spent by then government is then deposited by recipients into commercial banks. It is the banks' liability. But as long as they are on hand, it is also banks' asset. Again, the accounting book of commercial banks is balanced.

Through the magic of fractional-reserve banking, 90\% (assumed) of the deposit in the commercial banks is available for lending. How can this money be lent out when it is owned by the original depositors who are still free to write checks? The answer is that new money is created when banks make loans. The loans then come back to the banks and become deposit which is available for lending. Let's consider an example, \$1,000,000 gives birth to \$900,000. \$900,000 gives birth to \$810,000. The process goes forever. One may calculate the theoretical maximum of money created by summing an infinite geometric sequence.
\end{document}
